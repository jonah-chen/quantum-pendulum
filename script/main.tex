\documentclass[a4paper]{article}
\usepackage[margin=1in]{geometry}
\usepackage{listings}
\usepackage[utf8]{inputenc} % allow utf-8 input
\usepackage[T1]{fontenc}    % use 8-bit T1 fonts
\usepackage{hyperref}       % hyperlinks
\usepackage{url}            % simple URL typesetting
\usepackage{booktabs}       % professional-quality tables
\usepackage{amsfonts}       % blackboard math symbols
\usepackage{nicefrac}       % compact symbols for 1/2, etc.
\usepackage{microtype}      % microtypography
\usepackage{lipsum}     % Can be removed after putting your text content
\usepackage{graphicx}
\usepackage{titlesec}
\usepackage{fancyhdr}
\usepackage{siunitx}
\usepackage{amsmath}
\usepackage{physoly}
\pagestyle{fancy}
\numberwithin{equation}{section}
\titleformat{\section}
  {\large\scshape}{\thesection}{1em}{}
\titleformat{\subsection}
  {\normalfont\scshape}{\thesubsection}{1em}{}
\titleformat{\subsubsection}
  {\normalfont\scshape}{\thesubsubsection}{1em}{}
\pagenumbering{arabic}
\usepackage[
    backend=biber, 
    natbib=true,
    style=numeric,
    sorting=none
]{biblatex}
\usepackage{csquotes}
\addbibresource{citations.bib}

\setlength\parindent{0pt}

\title{\vspace{-50pt}\bfseries{\Large{Quirks of the Quantum Pendulum}}\vspace{-5pt}}
\author{\normalfont{Jonah Chen}\\\vspace{-3pt}\small{University of Toronto}\\
\vspace{-10pt}\small{\url{jonah.chen@mail.utoronto.ca}}}
\date{}

\begin{document}
\maketitle
\sffamily

Trying to make a three-blue-one-brown like video about the quantum harmonic oscillator, and many interesting properties that arise it if you just look slightly deeper than what is taught in a normal course.

\section{Introduction and Motivation}

\begin{script}
    What did you click on? We will talk about the quantum pendulum. I am a student and I have this taught to me multiple times. I hated how it was taught in class as it gave no insight whatsoever into the physics of the system, so in the spirit of three-blue-one-brown, I will teach a different way, a method usually taught in more advanced quantum mechanics courses, but hopefully digestable by a slightly above-average high school student. By the end of the video, I hope to explain how the pendulum actually oscillates, and how it is related to a completely different process involving the roulette wheel or radioactive materials, and also how we can saturate the heisenberg uncertainty principle to build lasers.
\end{script}

\begin{script}
    I decided to split the video into multiple segments. First, I will introduce the mathematical preliminaries since the maths required for our journey is quite complex. If you are strong at calculus and linear algebra, feel free to skip this section. Then, we will find the solutions to the quantum harmonic oscillator without having to solve the schrodinger's equation at all. Finally, from the insights we have gained, we will discuss uncover several interesting properties this system has and some of its applications to modern physics and engineering, which could seem opaque at the surface.
\end{script}

\subsection{Prereqs}
\begin{itemize}
    \item Basic calculus and linear algebra
\end{itemize}

\section{Mathematical Preliminaries}

\subsection{Linear Operators}

\subsection{Inner Product and Bra-Ket}

\subsection{Hermitian Operators, Adjoints, and Spectra}

% MUST mention Spectral Theorem
\subsection{Schrodinger's Equation}
Usually students see the Schrodinger's equation presented in two forms, the time independent equation which is
\begin{align}
    E\ket\psi&=\hat H\ket\psi\\\label{time_indep_SE}
    &=\left(\frac{\hat p^2}{2m}+V(\hat x)\right)\ket\psi\\
    &=\left(-\frac{\hbar^2}{2m}\partial_x^2+V(x)\right)\ket\psi
\end{align}
These three equations are equivalent, as the RHS is just expanding what the hamiltonian \(\hat H\) is with different specificities. Note that this looks like an eigenvalue equation because it is one, and this equation is also called the ``energy eigenvalue equation''. The time dependent equation is
\begin{align}
    i\hbar\partial_t\ket\psi=\hat H\ket\psi\label{time_dep_SE}
\end{align}
This equation actually governs how quantum systems evolve in time, similar to how Newton's or Euler-Lagrange's equation govern how classical systems evolve in time.
\begin{align}
    -\grad V&=\derivative{\mathbf p}{t}\\
    \partialderivative{L}{q_i}+\derivative{}{t}\left(\partialderivative{L}{\dot q_i}\right)&=0
\end{align}
First, notice that the schrodinger equation is a linear equation. This means given two solutions, any linear combination of them is also a solution. Second, the time dependent equation can be solved easily for solutions of the time independent equation. For the initial state \(\ket{E;0}\) such that \(E\ket{E;0}=\hat H\ket{E;0},\)
\begin{align}
    i\hbar\partial_t\ket{E;t}&=\hat H\ket{E;t}\\
    &=E\ket{E;t}\\
    \ket{E;t}&=e^{-iEt/\hbar}\ket{E;0}
\end{align}
Due to the spectral theorem, any state of the system can be written as a linear combination of energy eigenstates, either of the following depending on if the system has a discrete or continuous spectrum.
\begin{align}\label{sln_bound}
    \ket{\psi;t} &= \sum_{n=0}^\infty\ket{E_n;t}=\sum_{n=0}^\infty e^{-iE_nt/\hbar}\ket{E_n;0}\\
    \ket{\psi;t} &= \int_{-\infty}^\infty\dd E\ket{E;t} = \int_{-\infty}^\infty\dd E\,e^{-iEt/\hbar}\ket{E;0}\label{sln_scattering}
\end{align}
Note that there was no assumption made apart from the fact we know the solution of the time independent equation. Hence, this solution is general. Meaning, just by solving the time independent equation, we can fully determine how any state of the system evolves in time. From now on, (for a long time), we will only discuss the time independent wavefunction \(\psi(x)=\braket{x}{\psi}.\)

You may wonder why there are two different equations. Well, that has to do with if the state described is a bound or scattering state.

If it was a bound state, that means the particle is bound to the potential. Meaning, as \(x\to\pm\infty, E<V(x).\) Let's isolate the second derivative of the wavefunction
\begin{equation}
    \partial^2_x\psi(x)=\frac{2m}{\hbar}(V(x)-E)\psi(x)=k\psi(x)
\end{equation}
(Very Loosely) Here \(k>0\). Let's say the wave function is positive as \(x\to\infty,\) then the second derivative is positive. This means the wavefunction is curving up. Hence, the energy \(E\) must take discrete values for the wave function to be in \(L^2.\) Otherwise, just changing the energy very slightly will cause the wave function to blow up.

This is not true for scattering states where \(k<0,\) as when the wavefunction is positive the second derivative is negative, causing the wavefunction to curve down and oscillate. Hence, a continuous spectrum of energies is valid as scattering states.

\section{Solving the Quantum Pendulum}

We will try to solve for the energies and wavefunctions of the quantum harmonic oscillator. The hamiltonian is given by
\begin{align}
    \hat H&=\frac{\hat p^2}{2m}+\frac{1}{2}m\omega^2\hat x^2\label{qho_hamiltonian}
\end{align}

\subsection{Dimensionless Position and Momentum}

It is common in physics to solve a problem by using quantities that are natural to the problem. By using quantities that are natural, like the mass of the particle \(m,\) the frequency of the oscillator \(\omega,\) and one of the fundamental constants for quantum mechanics \(\hbar,\) to define more natural position and momentum coordinates as
\begin{align}
    X=x\sqrt{\frac{m\omega}{2\hbar}}
\end{align}
Then, the operators
\begin{align}
    \hat X &= \hat x\sqrt{\frac{m\omega}{2\hbar}}=X\\
    \hat P &= \hat p\sqrt{\frac{1}{2m\hbar\omega}}=i\partial_X
\end{align}

Substituting these natural coordinates for \(\hat x\) and \(\hat p\) in equation \eqref{qho_hamiltonian} yields
\begin{align}
    \hat H&=\frac{\left(2m\hbar\omega\hat P^2\right)}{2m}+\frac{1}{2}m\omega^2 \left(\frac{2\hbar}{m\omega}\hat X^2\right)\\
    &=\hbar\omega\left(\hat P^2+\hat X^2\right)
\end{align}

\subsection{Brute Force Approach}
\begin{script}
    This is how I was taught in school two or three times. It is quite boring so I will go through it in a minute. There is no need to understand any of the math here, as they would be irrelevent for our understanding of the physical system. 
\end{script}
As we established, we only need to solve the time independent schrodinger equation. With the harmonic oscillator hamiltonian, we can write the schrodinger equation as
\begin{align}
    -\derivative{^2\psi(X)}{X^2}+X^2\psi(X)=E\psi(X)
\end{align}
Now, we evaluate the behavior as \(X\to\pm\infty.\) Then, \(X^2>>E\) so the term \(E\psi(X)\) can be neglected. The resulting equation is
\begin{align}
    -\derivative{^2\psi(X)}{X^2}+X^2\psi(X)&=0\\
    \derivative{^2\psi(X)}{X^2}&=X^2\psi(X)
\end{align}
For \(\psi(X)=e^{-X^2/2},\) the second derivative is \((X^2+1)\psi(X),\) which for \(X\to\pm\infty\) is close to \(X^2\psi(X)\). So, from this we guess the solution is in the form \(\psi(X)=u(X)e^{-X^2/2}\). Substituting into the equation yields
\begin{align}
    \derivative{}{}
\end{align}
\subsection{Sum and Difference of Squares}
We've all learned in grade school that a difference of squares, \(a^2-b^2\), can be factored into \(a-b\) and \(a+b.\) In a similar manner, the sum of squares \(a^2+b^2\) can be factored into \(a+ib\) and \(a-ib\). We can prove this easily by using the distributive property of multiplication and the fact that \(i^2=-1.\)
\begin{align}
    (a-ib)(a+ib)&=a^2+b^2+\cancel{(a)(ib)}+\cancel{(-ib)(a)}\label{sum_squares_commute}
\end{align}
If \(a\) and \(b\) are real or complex numbers, this is perfectly fine as the order which you multiply \(a\) and \(b\) does not matter. But this is a problem when \(a\) and \(b\) are linear operators, as they may not commute. If we can't assume \(ab=ba,\) equation \eqref{sum_squares_commute} is false. Instead, we can define the \textbf{commutator} as how different \(ab\) is from \(ba,\) as
\begin{equation}
    [a,b]=:ab-ba
\end{equation}
So this means,
\begin{align}
    (a-ib)(a+ib)&=a^2+b^2+(a)(ib)+(-ib)(a)\\
    &=a^2+b^2+i(ab-ba)\\
    &=a^2+b^2+i[a,b]\label{sum_squares}
\end{align}
As an example, we will compute the commutator between \(\hat X\) and \(\hat P.\)
\begin{align}
    [\hat X,\hat P]&=\hat X\hat P-\hat P\hat X\\
    &=\sqrt{\frac{m\omega}{2\hbar}}\hat x\sqrt{\frac{1}{2m\hbar\omega}}\hat p-\sqrt{\frac{1}{2m\hbar\omega}}\hat p\sqrt{\frac{m\omega}{2\hbar}}\hat x\\
    &=\frac{1}{2\hbar}\left(\hat x\hat p-\hat p\hat x\right)\\
    &=\frac{1}{2\hbar}\left(x(i\hbar\partial_x)-(i\hbar\partial_x)x\right)\\
    &=\frac{i}{2}(x\partial_x-\partial_xx)
\end{align}
It may be confusing how to compute this quantity, but note what linear operators do is act on functions. The sum or product of linear operators is still a linear operator, thus this commutator should be a linear operator. Hence, let's try to compute the commutator on an arbiturary function \(f,\) using the product rule:
\begin{align}
    (x\partial_x-\partial_xx)f&=x\partialderivative{f}{x}-\partial_x(xf(x))=x\partialderivative{f}{x}-f-x\partialderivative{f}{x}=-f\\
    (x\partial_x-\partial_xx)&=-\mathbb{I}
\end{align}
So,
\begin{align}
    [\hat X,\hat P]&=-\frac{i}{2}\mathbb{I}
\end{align}
and
\begin{align}
    \hat X^2+\hat P^2&=(\hat X-i\hat P)(\hat X+i\hat P)+i[\hat X,\hat P]\\
    &=(\hat X-i\hat P)(\hat X+i\hat P)+\frac{1}{2}\mathbb{I}
\end{align}
From now on, we will not write out the identity operator explicitly.

\noindent We define the two factors as the annihilation operator \(\hat a\) and the creation operator \(\hat a^\dagger\).
\begin{align}
    \hat a &= X+i\hat P\\
    \hat a^\dagger &= X-i\hat P
\end{align}
And we can write the hamiltonian as
\begin{align}
    \hat H=\hbar\omega\left(a^\dagger a+\frac{1}{2}\right)
\end{align}
I wonder where the ground state energy of \(\frac{1}{2}\hbar\omega\) comes from.
\subsection{Solving the Problem the ``Proper'' Way}
We have to calculate the commutation relations
\begin{align}
    [\hat a,\hat a^\dagger]&=[(\hat X-i\hat P),(\hat X+i\hat P)]\\
    &=[\hat X, \hat X]+[\hat X,i\hat P]+[-i\hat P,\hat X]+[i\hat P,-i\hat P]\\
    &=0+\frac{1}{2}+\frac{1}{2}+0\\
    &=1
\end{align}
For convenience, we will define a number operator
\begin{align}
    \hat N &= \hat a^\dagger\hat a
\end{align}
Here is the solution to the problem. Assume we have some vector \(\ket\psi\) that is an eigenvector of \(\hat N\) with eigenvalue \(n.\) Then, 
\begin{align}
    \hat H\ket\psi&=\hbar\omega\left(\hat N+\frac{1}{2}\right)\ket\psi\\
    &=\hbar\omega\left(n+\frac{1}{2}\right)\ket\psi
\end{align}
so \(\ket\psi\) is also be an eigenvector to \(\hat H\) with eigenvalue \(E=\hbar\omega\left(n+\frac{1}{2}\right).\) Now we consider \(\hat a\ket\psi\) and \(\hat a^\dagger\ket\psi.\) Are these also eigenvectors? If so, what would be the eigenvalues?
\begin{align}
    \hat N(\hat a\ket\psi)&=\hat a^\dagger\hat a\hat a\ket\psi\\
    &=\hat a(\hat a^\dagger\hat a-[\hat a,\hat a^\dagger])\ket\psi\\
    &=\hat a(\hat N-1)\ket\psi\\
    &=(n-1)(\hat a\ket\psi)\\
    \hat H(\hat a\ket\psi)&=(E-\hbar\omega)\ket\psi
\end{align}
So, \(\hat a\ket\psi\) is also an eigenvector, but the energy eigenvalue is decreased by \(\hbar\omega.\)
\begin{align}
    \hat N(\hat a^\dagger\ket\psi)&=\hat a^\dagger\hat a\hat a^\dagger\ket\psi\\
    &=\hat a^\dagger(\hat a^\dagger\hat a+[\hat a,\hat a^\dagger])\ket\psi\\
    &=\hat a^\dagger(\hat N+1)\ket\psi\\
    &=(n+1)(\hat a^\dagger\ket\psi)\\
    \hat H(\hat a^\dagger\ket\psi)&=(E+\hbar\omega)\ket\psi
\end{align}
So, \(\hat a^\dagger\ket\psi\) is also an eigenvector, but the energy eigenvalue is increased by \(\hbar\omega.\)

\noindent We know that at least one \(\ket\psi\) exists because of the E\&U theorems. However, we know that the physical system must have a minimum energy. Let's call this state \(\ket0\). Then, we know this state must be the kernel of \(\hat a\), as otherwise \(\hat a\ket0\) would have lower energy that \(\ket0.\) The ground state energy of the harmonic oscillator is now trivial,
\begin{align}
    \hat H\ket 0=\hbar\omega\left(\hat a^\dagger\hat a+\frac{1}{2}\right)\ket 0=\frac{1}{2}\hbar\omega\ket0
\end{align}
We can write the annihilation operator in the
\begin{align}
    \hat{a}=\left(X+\partial_X\right)
\end{align}
Let \(\braket{x}{0}=\psi_0(x)\)
\begin{align}
    \hat a\ket0&=(X+\partial_X)\psi_0(X)=0\\
    \partialderivative{\psi_0}{X}&=-X\psi_0(X)\\
    \psi_0(X)&=Ae^{-X^2/2}
\end{align}
Using the normalization condition \(\braket{0}{0}=1,\) we can find \(A=1/\sqrt{2\pi}.\) This is the exact same solution as the previous section.


\section{Implications}

\subsection{Exponentials and Translations}

\subsection{Poisson Distribution and Coherent States}

\subsection{Uncertainty and Squeezed States}

\begin{thebibliography}{99}

\end{thebibliography}

\end{document}